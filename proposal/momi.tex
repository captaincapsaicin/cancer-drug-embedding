\section{Momi results}

Including the Hadza into Xoo pulse improves log likelihood substantially ($\sim$40k units) over the no pulse model. The pulse proportion is also very high. I emphasize that the other timings are very unstable, however, and will differ wildly from run to run. Momi results are from setting San population sizes for times $>$100 kya, and using a subsample of the data with 3 individuals from each population (larger samples take too many computational resources).

Heartening results from momi include:
\begin{enumerate}
\item Allowing for a double pulse between Hadza and San shows strong directionality \ref{fig:double}
\item Allowing for ancient ghost admixture into San does not dull signal from Hadza into San \ref{fig:ghost}
\item Allowing for a pulse from San into Sandawe does not show a strong signal \ref{fig:sandawe}
\end{enumerate}


\begin{figure}[h!]
\includegraphics[width=\textwidth]{figs/momi_hadza_sandawe_double.png}
\caption{Allowing for a double pulse between Hadza and San shows strong directionality}
\label{fig:double}
\end{figure}


\begin{figure}[h!]
\includegraphics[width=\textwidth]{figs/momi_ancient_hadza.png}
\caption{Allowing for ancient ghost admixture into San does not dull signal from Hadza into San}
\label{fig:ghost}
\end{figure}

\begin{figure}[h!]
\includegraphics[width=\textwidth]{figs/momi_xoo_sandawe.png}
\caption{Allowing for a pulse from San into Sandawe leads to a $~$7k improvement in log likelihood over the no-pulse model. In addition, across multiple runs, the pulse timing doesn't take advantage of differentiation between Hadza and Sandawe}
\label{fig:sandawe}
\end{figure}
