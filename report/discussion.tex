\section{Discussion}

The \textit{graph2vec} embedding, qualitatively, shows promise for learning structural motifs from large unlabeled molecular datasets. In addition, there seems to be some nontrivial correlation between properties of chemical interest and their positioning within the embeddings.

Possible applications of this kind of embedding include refining molecular searches with queries of the type ``find a compound with structural similarity to compound X, while maximizing predicted probability of activity Y.'' In addition, molecular embeddings serve as a superior alternative to molecular kernels (some of which are based on the Weisfeiler Lehman kernel), as one can train arbitrary machine learning models on top of them for any type of supervised problem (as opposed to only kernel-based methods like SVMs). Future work should involve using these embeddings for prediction tasks on labeled datasets, which will be easier in the future as hopefully more datasets become open.

In particular, the work described in this paper suffers from a major flaw, which is that it does not take into account node labels within the graph. A path between C-C and N-C, say, are treated as identical, as long as the connectivity of each node is the same. Including element labels in the \textit{graph2vec} subgraph vocabulary construction should be a first priority in future work, and should improve any prediction task based on the embedding. In addition, the application of \textit{graph2vec} described here does not take into account edge labels (e.g. bond type), so single, double, and triple bonds are not explicitly treated differently. The implicit change in connectivity of neighboring molecules would give them different signatures, but this is not always enough, for example in the graphs C-C and C=C. An explicit consideration of distinct bond types could improve molecular embeddings, as different bond types have different 3 dimensional geometry. Another avenue would be to include bond information implicitly by explicilty including hydrogens in the construction of the molecular graph, guaranteeing that there would be downstream connectivity changes for a bond type change.

Much of the future of cheminformatics depends on improved predictors for complex behavior, especially for drug screening, where patient response can depend on a myriad of biological pathways. Leveraging rich labeled datasets of compound response, like those produced by high-throughput light microscopy drug screens in Kang et al \cite{Kang2016} to improve compound embeddings is another line of future work.

